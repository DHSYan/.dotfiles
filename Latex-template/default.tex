\documentclass[12pt]{article}
\usepackage[top=0.5in, bottom=1in, left=1in, right=1in]{geometry}
\usepackage{graphicx}
\usepackage{listings}
\usepackage{hyperref}
\usepackage{fancyvrb}
\usepackage{verbatim}
%%%%%%%%%%%%%%%%%%%%%%%%%%%Class Premble%%%%%%%%%%%%%%%%%%%%%%%%%%%%%%%%%%%%%%%%
\usepackage{listings}
\usepackage{xcolor}

\definecolor{codegreen}{rgb}{0,0.6,0}
\definecolor{codegray}{rgb}{0.5,0.5,0.5}
\definecolor{codepurple}{rgb}{0.58,0,0.82}
\definecolor{backcolour}{rgb}{0.95,0.95,0.92}

\lstdefinestyle{mystyle}{
    backgroundcolor=\color{backcolour},   
    commentstyle=\color{codegreen},
    keywordstyle=\color{magenta},
    numberstyle=\tiny\color{codegray},
    stringstyle=\color{codepurple},
    basicstyle=\ttfamily\footnotesize,
    breakatwhitespace=false,         
    breaklines=true,                 
    captionpos=b,                    
    keepspaces=true,                 
    numbers=left,                    
    numbersep=5pt,                  
    showspaces=false,                
    showstringspaces=false,
    showtabs=false,                  
    tabsize=2
}

\lstset{style=mystyle}

%%%%%%%%%%%%%%%%%%%%%%%%%%%%%%%%%%%%%%%%%%%%%%%%%%%%%%%%%%%%%%%%%%%%%%%%%%%%%%%%

%%%%%%%%%%%%%%%%%%%%%%%%%%%%%%%Commands%%%%%%%%%%%%%%%%%%%%%%%%%%%%%%%%%%%%%%%%%
\newcommand{\lil}[1]{\lstinline{#1}}

%%%%%%%%%%%%%%%%%%%%%%%%%%%%%%%%%%%%%%%%%%%%%%%%%%%%%%%%%%%%%%%%%%%%%%%%%%%%%%%%

\begin{document}
\noindent
\textbf{COMP3331\hfill UNSW\\Lab 3\hfill Sean Yan}

\section*{Exercise 3}
\begin{enumerate}

    \item The IP of www.stanford.edu is 151.101.30.133. And the type of DNS query is 
        the record type A. 

    \item The canonical name for the Stanford webserver is 
        pantheon-systems.map.fastly.net. For Stanford 
        specifically, fastly.net is a web hosting provider.

    \item From the other fields of the dig output. We can infer that 
        a cookie ID was being returned. We are using the UDP protocol. 
        Using DNS Version 0. There was no error in the status returned. We 
        are using the Operation: "QUERY". And we are querying DNS-Record type
        A. Lastly, our query took 40ms. The DNS Server that is used for 
        this query is 129.94.242.2, which by doing a reverse DNS lookup,
        we can see that this DNS Server belongs and is in the 
        UNSW Network. We can also see a timestamp of this query and the size
        of the query.

    \item My local machine in my home has the DNS Server 192.168.0.1. \\
        The machine in VLAB has the DNS Server 129.94.242.2. \\ 
        These values where obtained through the \lstinline{dig} command.\\
        \textbf{Output: }
        \verbatiminput{output-files/q10-0}

    \item argus.stanford.edu - 171.64.7.115 \\
        ns7.dnsmadeeasy.com - 208.80.126.13 \\
        ns5.dnsmadeeasy.com - 208.94.148.13 \\
    	ns6.dnsmadeeasy.com - 208.80.124.13 \\
        avallone.stanford.edu - 204.63.224.53 \\
        atalante.stanford.edu - 171.64.7.61 \\ 
        Are the name server for stanford.edu 

        The DNS query type is \lstinline{NS}.\\
        \textbf{Output: }
        \verbatiminput{output-files/dig-output-stanford}
        \verbatiminput{output-files/dig-output-stanford-type-ns}

    \item ece.drexel.edu is the DNS Name associated with 129.25.60.56. The 
          \lstinline{PTR} is used to obtain this information\\ 

          \textbf{Output: }
          \verbatiminput{output-files/dig-output-q6}

    \item The mail server is \lstinline{smtp.google.com}. The output did not
        give an authoritative answer. In the flags section given by 
        \lstinline{dig} did not show the flag \lstinline{aa} and it
        instead shows \lstinline{ra} which means recursive available. Moreover, 
        the output states the answer to our query was responaded by the same
        server we queried, which is the CSE UNSW Nameserver.
 
        \textbf{Output: }
        \verbatiminput{output-files/dig-output-q7}

    \item The query did not get an answer. The recursion request was not 
        available. And the server returned the REFUSED status.\\

        \textbf{Output:}
        \verbatiminput{output-files/dig-output-q8}

    \item To get an authoritative answer, we should obtain google's 
        authoritative Name server, by using \lstinline{dig} with the type 
        \lstinline{NS}. And querying using the resulting name server, with 
        thi \lstinline{MX} type. By looking at the flags returned by 
        \lstinline{dig}, we see an \lstinline{aa} tag, which proves that 
        the answer is an authoritative answer.

        \textbf{Output: }
        \verbatiminput{output-files/output-q9-1}
        \verbatiminput{output-files/output-q9-2}

    \item First, the IP Address of \lil{lyre00.cse.unsw.edu.au} is 
        $129.94.210.20$. 
        \verbatiminput{output-files/q10-0}
        Next, we find the \lil{.} domain's Name Server. 
        \verbatiminput{output-files/q10-1}
        We will query \lil{a.root-servers.net.} to find \lil{au.}'s name server.
        \verbatiminput{output-files/q10-2}
        Querying \lil{q.au} to find \lil{edu.au.}'s Name Server.
        \verbatiminput{output-files/q10-3}
        Again, querying \lil{s.au} to find \lil{unsw.edu.au}'s Name Server.
        \verbatiminput{output-files/q10-4}
        Then, querying \lil{ns1.unsw.edu.au} to find \lil{cse.unsw.edu.au}'s 
        name server.
        \verbatiminput{output-files/q10-5}
        Finally, querying \lil{beethoven.orchestra.cse.unsw.edu.au.}
        to find \lil{lyre00.cse.unsw.edu.au}'s \lil{A} Record.\\ 
        Giving us, 
        \verbatiminput{output-files/q10-6}

        It took 5 DNS servers to get this authoritative answer.

    \item Yes, in the process of doing question 10 of exercise 3.
      I found out that \lil{beethoven.orchestra.cse.unsw.edu.au} has 
      at least 3 IP Associated with it. Therefore, one machine can definitely
      have multiple IP Address. In terms of multiple names, inserting a 
      \lil{CNAME} record into a DNS Servers, will accomplish this.
      \verbatiminput{output-files/q10-5}
\end{enumerate}

\newpage

\section*{Exercise 4}
Result:
\begin{center}
    \includegraphics[scale=0.2]{img/2023-10-09-09-41-42.png}\\
    \caption{}
\end{center}
\begin{center}
    \includegraphics[scale=0.2]{img/2023-10-09-09-42-45.png}\\
  \caption{}
\end{center}
\begin{center}
    \includegraphics[scale=0.2]{img/2023-10-09-09-43-51.png}\\
  \caption{}
\end{center}
The implementation of sending images over http was not sucessful.

%%%%%%%%%%%%%%%%%%%%%%%%%%%%%%%%%%%%%%%%%%%%%%%%%%%%%%%%%%%%%%%%%%%%%%%%%%%%%%%%
\footnotetext{Lab was completed by Sean Yan on \today}


\newpage

\section*{References}
% This lab report's work was inspired by the following resources,
\newcommand{\link}[1]{\href{#1}{\lil{#1}}}
\begin{enumerate}
    \item [1] I searched wikipedia to get a general sense of why CNAME aliases 
        are used in Networking. \\
        \link{https://en.wikipedia.org/wiki/CNAME_record}

    \item [2] I searched online for the meaning for the flags in a dig response. \\
        \link{https://serverfault.com/questions/729025/what-are-all-the-flags-in-a-dig-response}

    \item [3] For exercise 4: I have used Beej's guide to help me understand 
        the library that is used to construct servers and other general
        understanding of socket programming\\
        \link{https://beej.us/guide/bgnet/html/}\\
        \link{https://www.binarytides.com/socket-programming-c-linux-tutorial/}\\
        \link{http://www.tutorialspoint.com/unix_sockets/index.htm}

    \item [4] to answer the question: Why does "getaddrinfo()" return a LinkedList \\
        \link{https://www.ibm.com/docs/en/zos/2.4.0?topic=functions-getaddrinfo-get-address-information}

    \item [5] A YouTube video that help me understand the flow of the server's
        behavior. \\
        \link{https://youtu.be/esXw4bdaZkc?si=XgIewcOKpFbSvgAW}

    \item [6] Referenced to contruct HTTP Response Message. \\
        \link{https://developer.mozilla.org/en-US/docs/Web/HTTP/Messages}

    \item [7] how to read a content of file into C string\\ 
        \link{https://stackoverflow.com/questions/174531/how-to-read-the-content-of-a-file-to-a-string-in-c}\\
        \link{https://www.tutorialspoint.com/cprogramming/c_file_io.htm}\\
        \link{https://www.geeksforgeeks.org/c-program-to-read-contents-of-whole-file/}

    \item [8] String Parsing Inspiration\\
        \link{https://medium.com/@iiesbangalorebl1/how-to-implement-string-parsing-in-c-language-a-comprehensive-guide-6bf5a1c4a214}

    \item [9] scanning current directory for files \\
        \link{https://stackoverflow.com/questions/8149569/scan-a-directory-to-find-files-in-c}\\
        \link{https://pubs.opengroup.org/onlinepubs/009604599/functions/readdir.html}\\
        \link{https://en.wikibooks.org/wiki/C_Programming/POSIX_Reference/dirent.h}

    \item [10] Researched \lil{restrict}'s semantics in C\\
        \link{https://en.wikipedia.org/wiki/Restrict}

    \item [11] Reference C Documentation.\\
        \link{https://cplusplus.com/}\\
        \link{https://pubs.opengroup.org/onlinepubs/007908799/xns/netdb.h.html}
        \link{https://pubs.opengroup.org/onlinepubs/007908799/xns/netinetin.h.html}

    \item [12] Referenced course code example on TCPServer in C \\
        \link{https://webcms3.cse.unsw.edu.au/COMP3331/23T3/resources/90303}

    \item [13] For the failed implementation of sending images.\\
        \link{https://stackoverflow.com/a/24893435}\\
        \link{https://www.w3schools.com/tags/tag_img.asp}





\end{enumerate}

\end{document}



